\documentclass[11pt]{article}

% ------------------------------------------------------------------------------
% This is all preamble stuff that you don't have to worry about.
% Head down to where it says "Start here"
% ------------------------------------------------------------------------------

\usepackage[margin=.8in,top=1.1in,bottom=1.1in]{geometry} % page layout
\usepackage{amsmath,amsthm,amssymb,amsfonts} % math things
\usepackage{graphicx} % include graphics
\usepackage{fancyhdr} % header customization
\usepackage{titlesec} % help with section naming
\usepackage{listings}
\usepackage[final]{pdfpages}

% naming sections
\titleformat{\section}{\bf}{Problem \thesection}{0.5em}{}
\newcommand{\exercise}{\section{}}

% headers
\pagestyle{fancy}
\fancyhf{} % clear all
\fancyhead[L]{\sffamily\small Machine Learning 1 --- Homework}
\fancyhead[R]{\sffamily\small Page \thepage}
\renewcommand{\headrulewidth}{0.2pt}
\renewcommand{\footrulewidth}{0.2pt}
\markright{\hrulefill\quad}

\newcommand{\hwhead}[4]{
\begin{center}
\sffamily\large\bfseries Machine Learning Worksheet #1
\vspace{2mm}
\normalfont

#2\\
#3\\
\texttt{#4}
\end{center}
\vspace{6mm} \hrule \vspace{4mm}
}

% ------------------------------------------------------------------------------
% Start here -- Fill in your name, imat and email
% (and the same for who you worked with)
% You are allowed to work in groups of 2 (or 3 if there is no way around it)
% However, you each must submit individually - (it may be same file)
% ------------------------------------------------------------------------------

\newcommand{\names}{Tomas Ladek, Michael Kratzer} %
\newcommand{\imats}{3602673, } %
\newcommand{\emails}{tom.ladek@tum.de, } %

\begin{document}

% ------------------------------------------------------------------------------
% Change xx (and only xx) to the current sheet number
% ------------------------------------------------------------------------------
\hwhead{3}{\names}{\imats}{\emails}

% ------------------------------------------------------------------------------
% Fill in your solutions
% ------------------------------------------------------------------------------

\exercise
\begin{align*}
	\dfrac{\partial}{\partial \theta}\theta^t (1-\theta)^h &= t \theta^{t-1} (1-\theta)^h - \theta^t h(1-\theta)^{h-1}\\
	&= \theta^{t-1}(1-\theta)^{h-1}\big(t(1-\theta)-h\theta\big)
\end{align*}

\begin{align*}
	\dfrac{\partial^2}{\partial \theta^2}\theta^t (1-\theta)^h &= \dfrac{\partial}{\partial \theta}\theta^{t-1}(1-\theta)^{h-1}\big(t(1-\theta)-h\theta\big)\\
	 &= -2\,t\,h\,\theta^{t-1}(1-\theta)^{h-1} + t(t-1)\theta^{t-2}(1-\theta)^h + h(h-1)\theta^t(1-\theta)^{h-2}
\end{align*}

\begin{align*}
	\dfrac{\partial}{\partial \theta}\log\theta^t (1-\theta)^h &= \dfrac{\partial}{\partial \theta}\big(t\log\theta + h\log(1-\theta)\big)\\
	&= \dfrac{t}{\theta} - \dfrac{h}{1-\theta}
\end{align*}

\begin{align*}
	\dfrac{\partial^2}{\partial \theta^2}\log\theta^t (1-\theta)^h &= \dfrac{\partial}{\partial \theta}(\dfrac{t}{\theta} - \dfrac{h}{1-\theta})\\
	&= \dfrac{h}{(1-\theta)^2} - \dfrac{t}{\theta^2} 
\end{align*}

\exercise
The local extremum of $\log f(\theta)$ can determined by setting the first derivative to zero, i.e.
\begin{align*}
	\dfrac{d}{d \theta}\log\big(f(\theta)\big) \stackrel{!}{=} 0
\end{align*}
Evaluating this with the logarithm derivative rule leads to 
\begin{align*}
	\dfrac{\dfrac{d}{d\theta}f(\theta)}{f(\theta)} = 0 &\iff \dfrac{d}{d\theta}f(\theta) = 0
\end{align*}
Which is also the formula for the extremum of a differentiable positive function $f(\theta)$. That means that taking the logarithm of a function preserves its extremum points (in particular its local maximum points).\\
Considering the results from the first exercise, it can be easier to differentiate the logarithm of a function instead of the function itself, when one is interested only in the location of the maximum and not its value.

\exercise

\end{document}