\documentclass[11pt]{article}

% ------------------------------------------------------------------------------
% This is all preamble stuff that you don't have to worry about.
% Head down to where it says "Start here"
% ------------------------------------------------------------------------------

\usepackage[margin=.8in,top=1.1in,bottom=1.1in]{geometry} % page layout
\usepackage{amsmath,amsthm,amssymb,amsfonts} % math things
\usepackage{graphicx} % include graphics
\usepackage{fancyhdr} % header customization
\usepackage{titlesec} % help with section naming
\usepackage{listings}

% naming sections
\titleformat{\section}{\bf}{Problem \thesection}{0.5em}{}
\newcommand{\exercise}{\section{}}

% headers
\pagestyle{fancy}
\fancyhf{} % clear all
\fancyhead[L]{\sffamily\small Machine Learning 1 --- Homework}
\fancyhead[R]{\sffamily\small Page \thepage}
\renewcommand{\headrulewidth}{0.2pt}
\renewcommand{\footrulewidth}{0.2pt}
\markright{\hrulefill\quad}

\newcommand{\hwhead}[4]{
\begin{center}
\sffamily\large\bfseries Machine Learning Worksheet #1
\vspace{2mm}
\normalfont

#2\\
#3\\
\texttt{#4}
\end{center}
\vspace{6mm} \hrule \vspace{4mm}
}

% ------------------------------------------------------------------------------
% Start here -- Fill in your name, imat and email
% (and the same for who you worked with)
% You are allowed to work in groups of 2 (or 3 if there is no way around it)
% However, you each must submit individually - (it may be same file)
% ------------------------------------------------------------------------------

\newcommand{\names}{Tomas Ladek} %
\newcommand{\imats}{3602673} %
\newcommand{\emails}{tom.ladek@tum.de} %

\begin{document}

% ------------------------------------------------------------------------------
% Change xx (and only xx) to the current sheet number
% ------------------------------------------------------------------------------
\hwhead{2}{\names}{\imats}{\emails}

% ------------------------------------------------------------------------------
% Fill in your solutions
% ------------------------------------------------------------------------------

\exercise
After parsing the data that was given in a csv file and making up an efficient data structure (matrix), the Gini index of the root node ($C = \{0, 1, 2\}$) was calculated:
\begin{align*}
	i_G(t) &= 1 - (\dfrac{5}{15})^2 - (\dfrac{6}{15})^2 - (\dfrac{4}{15})^2
\end{align*}
Then in 0.1 steps from -0.6 to +10.0 (limits determined by data inspection), the Gini index of all possible left/right splits of the root node was calculated, for all features $x_{i,1}...x_{i,3}$. The maximum was a difference in Gini indices of $\approx0.3615$ for the first feature ($x_{i,1}$) for a split at the value 4.5\footnote{Due to the chosen calculation procedure, the split values take the maximum possible value between two different splits, instead of the average.}. The formula used for calculating the difference was

\begin{align*}
	\Delta i_G(t) (x_{i,1} \le s, t) &= i_G(t) - \dfrac{ \#classes\ in\ left\ tree}{\#classes\ in\ current\ data\ set}i_G(t_L) - \dfrac{ \#classes\ in\ right\ tree}{\#classes\ in\ current\ data\ set}i_G(t_R)
\end{align*}

with $i_G$ being the Gini indices of the current node, the left tree and the right tree respectively.\\

Splitting the root node at $x_{i,1} = 4.5$ yielded a left tree (values less than or equal to the split value) consisting of a pure node (class distribution of 100\% for class '1', Gini index 0) and a right tree combining the remaining classes '0' and '2', Gini index 0.4938. No further splits in the left tree were needed.\\

 Once again performing a maximalization on the Gini index differences for every possible split of every feature in 0.1 steps yielded feature $x_{i,1}$ at 7.4 as the best split. The result was a left sub-tree with class distribution '2': $\dfrac{2}{3}$; '0': $\dfrac{1}{3}$ and a right sub-tree with a pure distribution of class '0'. The corresponding Gini indices were 0.4444 and 0 respectively.\\
 
 As the maximum requested depth was reached, no further splitting was considered. The (raw) python code can be provided on request.
 
\end{document}