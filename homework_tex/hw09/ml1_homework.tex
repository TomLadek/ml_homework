\documentclass[11pt]{article}

% ------------------------------------------------------------------------------
% This is all preamble stuff that you don't have to worry about.
% Head down to where it says "Start here"
% ------------------------------------------------------------------------------

\usepackage[margin=.8in,top=1.1in,bottom=1.1in]{geometry} % page layout
\usepackage{amsmath,amsthm,amssymb,amsfonts} % math things
\usepackage{graphicx} % include graphics
\usepackage{fancyhdr} % header customization
\usepackage{titlesec} % help with section naming
\usepackage{listings}
\usepackage[final]{pdfpages}

% naming sections
\titleformat{\section}{\bf}{Problem \thesection}{0.5em}{}
\newcommand{\exercise}{\section{}}

% headers
\pagestyle{fancy}
\fancyhf{} % clear all
\fancyhead[L]{\sffamily\small Machine Learning 1 --- Homework}
\fancyhead[R]{\sffamily\small Page \thepage}
\renewcommand{\headrulewidth}{0.2pt}
\renewcommand{\footrulewidth}{0.2pt}
\markright{\hrulefill\quad}

\newcommand{\hwhead}[4]{
\begin{center}
\sffamily\large\bfseries Machine Learning Worksheet #1
\vspace{2mm}
\normalfont

#2\\
#3\\
\texttt{#4}
\end{center}
\vspace{6mm} \hrule \vspace{4mm}
}

% ------------------------------------------------------------------------------
% Start here -- Fill in your name, imat and email
% (and the same for who you worked with)
% You are allowed to work in groups of 2 (or 3 if there is no way around it)
% However, you each must submit individually - (it may be same file)
% ------------------------------------------------------------------------------

\newcommand{\names}{Tomas Ladek, Michael Kratzer} %
\newcommand{\imats}{3602673, 3612903} %
\newcommand{\emails}{tom.ladek@tum.de, mkratzer@mytum.de} %

\begin{document}

% ------------------------------------------------------------------------------
% Change xx (and only xx) to the current sheet number
% ------------------------------------------------------------------------------
\hwhead{9}{\names}{\imats}{\emails}

% ------------------------------------------------------------------------------
% Fill in your solutions
% ------------------------------------------------------------------------------

\exercise
Let us transform the row vectors of \cal{X} to column vectors so that we can input them directly into the covariance function:
\begin{align*}
	x_1 &= 
	\begin{pmatrix}
	0 \\
	1
	\end{pmatrix}\\
	x_2 &=
	\begin{pmatrix}
	0.5 \\
	2
	\end{pmatrix}
\end{align*}


\noindent Noise-free case means mean is zero: $m(x_*) = 0 = \mu_*$. Then:
\begin{align*}
	\boldsymbol{K} &= exp\big(-\dfrac{1}{2}(x - x')^T (x - x')\big)\\	
	  &= exp\big(-\dfrac{1}{2}
	  \begin{pmatrix}
		  x_1 - x_1 & x_1 - x_2 \\
		  x_2 - x_1 & x_2 - x_2
	  \end{pmatrix}^T 
	  \begin{pmatrix}
		  x_1 - x_1 & x_1 - x_2 \\
		  x_2 - x_1 & x_2 - x_2
	  \end{pmatrix}
	  \big)\\	  
	  &= exp\big(-\dfrac{1}{2}
	  \begin{pmatrix}
		  0 & -0.5 \\
		  0 & -1 \\
		  0.5 & 0 \\
		  1 & 0
	  \end{pmatrix}^T 
	  \begin{pmatrix}
		  0 & -0.5 \\
		  0 & -1 \\
		  0.5 & 0 \\
		  1 & 0
	  \end{pmatrix}
	  \big)\\	  
	  &= exp\big(-\dfrac{1}{2}
	  \begin{pmatrix}
		  1.25 & 0 \\
		  0 & 1.25
	  \end{pmatrix}
	  \big)\\	  
	  &=
	  \begin{pmatrix}
		  0.54 & 1 \\
		  1 & 0.54		  
	  \end{pmatrix}
\end{align*}

\begin{align*}
	\boldsymbol{K_*} &= exp\big(-\dfrac{1}{2}
	\begin{pmatrix}
		x_1 - x_* \\
		x_2 - x_*
	\end{pmatrix}^T 
	\begin{pmatrix}
		x_1 - x_* \\
		x_2 - x_*
	\end{pmatrix}
	\big)\\	
	&= exp\big(-\dfrac{1}{2}
	\begin{pmatrix}
		-0.5 & 0\\
		0 & 1
	\end{pmatrix}^T 
	\begin{pmatrix}
		-0.5 & 0\\
		0 & 1
	\end{pmatrix}
	\big)\\	
	&= exp\big(-\dfrac{1}{2}
	\begin{pmatrix}
		0.25 & 0 \\
		0 & 1
	\end{pmatrix}
	\big)\\	
	&=
	\begin{pmatrix}
		0.88 & 1 \\
		1 & 0.61
	\end{pmatrix} = \boldsymbol{K^T_*}
\end{align*}

\begin{align*}
	\boldsymbol{K_{**}} &= exp(-\dfrac{1}{2} (x_* - x_*)^T (x_* - x_*)) \\
		   &= exp(0) = 1
\end{align*}

Therefore
\begin{align*}
	f_{jt} =
	\begin{pmatrix}
		y_1\\
		y_2\\
		f(x_*)
	\end{pmatrix}
	\sim \mathcal N \Big(0, \Big[
	\begin{matrix}
		\boldsymbol{K} & \boldsymbol{K_*} \\
		\boldsymbol{K_*^T} & \boldsymbol{K_{**}}
	\end{matrix}
	\Big]\Big)
	= \mathcal N \Big(0,
	\begin{pmatrix}
		0.54 & 1 & 0.88 & 1\\
		1 & 0.54 & 1 & 0.61\\		
		0.88 & 1 & 1 & 0\\
		1 & 0.61 & 0 & 1
	\end{pmatrix}
	\Big)
\end{align*}

\exercise
?

\exercise
The joint distribution in the noisy case is given by
\begin{align*}
	\Big[
	\begin{matrix}
		\boldsymbol{y}\\
		f_*
	\end{matrix}
	\Big]
	\sim \mathcal{N} \Big(\Big[
	\begin{matrix}
		\mu\\
		\mu_*
	\end{matrix}
	\Big], \Big[
	\begin{matrix}
		\boldsymbol{K} + \sigma^2_n \boldsymbol{I} & \boldsymbol{K}_* \\
		\boldsymbol{K}^T_* & \boldsymbol{K}_{**}
	\end{matrix}
	\Big]
	\Big)
\end{align*}
which in our case evaluates to
\begin{align*}
	\mathcal N \Big(0,
	\begin{pmatrix}
		0.54 + \sigma^2_1 & 1 & 0.88 & 1\\
		1 & 0.54 + \sigma^2_2 & 1 & 0.61\\		
		0.88 & 1 & 1 & 0\\
		1 & 0.61 & 0 & 1
	\end{pmatrix}
	\Big)
\end{align*}

\end{document}