\documentclass[11pt]{article}

% ------------------------------------------------------------------------------
% This is all preamble stuff that you don't have to worry about.
% Head down to where it says "Start here"
% ------------------------------------------------------------------------------

\usepackage[margin=.8in,top=1.1in,bottom=1.1in]{geometry} % page layout
\usepackage{amsmath,amsthm,amssymb,amsfonts} % math things
\usepackage{graphicx} % include graphics
\usepackage{fancyhdr} % header customization
\usepackage{titlesec} % help with section naming
\usepackage{listings}
\usepackage[final]{pdfpages}

% naming sections
\titleformat{\section}{\bf}{Problem \thesection}{0.5em}{}
\newcommand{\exercise}{\section{}}

% headers
\pagestyle{fancy}
\fancyhf{} % clear all
\fancyhead[L]{\sffamily\small Machine Learning 1 --- Homework}
\fancyhead[R]{\sffamily\small Page \thepage}
\renewcommand{\headrulewidth}{0.2pt}
\renewcommand{\footrulewidth}{0.2pt}
\markright{\hrulefill\quad}

\newcommand{\hwhead}[4]{
\begin{center}
\sffamily\large\bfseries Machine Learning Worksheet #1
\vspace{2mm}
\normalfont

#2\\
#3\\
\texttt{#4}
\end{center}
\vspace{6mm} \hrule \vspace{4mm}
}

% ------------------------------------------------------------------------------
% Start here -- Fill in your name, imat and email
% (and the same for who you worked with)
% You are allowed to work in groups of 2 (or 3 if there is no way around it)
% However, you each must submit individually - (it may be same file)
% ------------------------------------------------------------------------------

\newcommand{\names}{Tomas Ladek, Michael Kratzer} %
\newcommand{\imats}{3602673, 3612903} %
\newcommand{\emails}{tom.ladek@tum.de, mkratzer@mytum.de} %

\begin{document}

% ------------------------------------------------------------------------------
% Change xx (and only xx) to the current sheet number
% ------------------------------------------------------------------------------
\hwhead{5}{\names}{\imats}{\emails}

% ------------------------------------------------------------------------------
% Fill in your solutions
% ------------------------------------------------------------------------------

\exercise
\begin{align*}
	E_{\cal D}(w) &= \dfrac{1}{2} \sum_{n=1}^N T_n [\mathbf{W}^T \phi(x_n) - Z_n]^2\\
	&= \dfrac{1}{2} [T (\Phi W - Z)^T] [T (\Phi W - Z)]
\end{align*}
with
\begin{align*}
	T &= 
	\begin{pmatrix}
	\sqrt{T_1} & & 0 \\
	 & \ddots & \\
	 0 & & \sqrt{T_n}
	\end{pmatrix}
\end{align*}
Now, finding the optimal $W$ so that this error function is minimal (using the knowledge about the derivative from the slides and the Matrix Cookbook):
\begin{align*}
	\nabla_W E_{\cal D}(W) &= \dfrac{\partial}{\partial W} \dfrac{1}{2} [T (\Phi W - Z)^T] [T (\Phi W - Z)] \\
	&= - T^2 \Phi^T (\Phi W - Z)
\end{align*}
Further, setting this to $0$ and solving for $W$:
\begin{align*}
	- T^2 \Phi^T (\Phi W - Z) &= 0 \\
	- T^2 \Phi^T \Phi W + T^2 \Phi^T Z &= 0 \\
	T^2 \Phi^T \Phi W &= T^2 \Phi^T Z \\
	(T^2 \Phi^T \Phi)^{-1} (T^2 \Phi^T \Phi W) &= (T^2 \Phi^T \Phi)^{-1} T^2 \Phi^T Z \\
	W &= T^{-2} T^2 (\Phi^T \Phi)^{-1} \Phi^T Z \\
	W &= (\Phi^T \Phi)^{-1} \Phi^T Z	
\end{align*}

By carefully choosing values of $T_n$, the error function can be made less sensitive against outliers (that can have a significant impact on the variance of the data noise). Also duplicate data points can be weighted accordingly (e.g. small value of $T_i$ for the data point copies: the error is also made smaller).

\exercise

\end{document}